\documentclass[conference]{IEEEtran}
\IEEEoverridecommandlockouts
% The preceding line is only needed to identify funding in the first footnote. If that is unneeded, please comment it out.
\usepackage{cite}
\usepackage{amsmath,amssymb,amsfonts}
\usepackage{graphicx}
\usepackage{textcomp}
\def\BibTeX{{\rm B\kern-.05em{\sc i\kern-.025em b}\kern-.08em
    T\kern-.1667em\lower.7ex\hbox{E}\kern-.125emX}}
\begin{document}

\title{THE1\\
}

\author{\IEEEauthorblockN{Esref Ozturk}
\IEEEauthorblockA{\textit{CENG} \\
\textit{METU}\\
Ankara, Turkey \\
esrefozturk93@gmail.com}
}

\maketitle

\begin{abstract}
This document is a model and instructions for \LaTeX.
This and the IEEEtran.cls file define the components of your paper [title, text, heads, etc.]. *CRITICAL: Do Not Use Symbols, Special Characters, Footnotes, 
or Math in Paper Title or Abstract.
\end{abstract}

\begin{IEEEkeywords}
component, formatting, style, styling, insert
\end{IEEEkeywords}

\section{Introduction}
This document is a model and instructions for \LaTeX.
Please observe the conference page limits. 

\section{Discussion}

\subsection{Effect of the Bias}

Accuracies with and without biases can be seen on the following table: \\

\begin{tabular}{l*{6}{c}r}
eta  & Accuracy(without bias) & Accuracy(with bias) \\
\hline
3e-4 & 0.4 & 0.48 \\
1e-3 & 0.57 & 0.58 \\
1e-1 & 0.65 & 0.79 \\
\end{tabular} \\

As it can be seen, bias increases the accuracy.

\subsection{Feature Scaling Technique}

Min-Max scaling technique is used for scaling each of the 8 features to 0-1 range. Using min-max scaling, all different range features are scaled to the same range. Luckily there was no feature that has the same max and min so min-max scaling were used.

\subsection{Effects of the Different Learning Rates}

Following table shows accuracy scores for different iteration counts: \\

\begin{tabular}{l*{6}{c}r}
eta  & 1000 & 10000 & 100000 \\
\hline
3e-4 & 0.37 & 0.48 & 0.61 \\
1e-3 & 0.37 & 0.58 & 0.73 \\
1e-1 & 0.73 & 0.79 & 0.79 \\
\end{tabular} \\

As iteration count increases, the gap between learning rates are decreasing. Small learning rates are causing the accuracy to increase very slowly. If we want to use small learning rates, we should increase the iteration count.

\subsection{Suggestion for a Better Learning Rate}

Using cross validation better learning rate can be selected. Also learning rate should be considered together with iteration count since for bigger iteration counts, smaller learning rates are enough.

\subsection{When to Stop Updates}

When to stop updates can be determined using cross validation. Iteration count can be parameterized for cross validation.

\subsection{Choice of Training and Validation Sets}






\end{document}
