\documentclass[conference]{IEEEtran}
\IEEEoverridecommandlockouts
% The preceding line is only needed to identify funding in the first footnote. If that is unneeded, please comment it out.
\usepackage{cite}
\usepackage{amsmath,amssymb,amsfonts}
\usepackage{graphicx}
\usepackage{textcomp}
\def\BibTeX{{\rm B\kern-.05em{\sc i\kern-.025em b}\kern-.08em
    T\kern-.1667em\lower.7ex\hbox{E}\kern-.125emX}}
\begin{document}

\title{THE1\\
}

\author{\IEEEauthorblockN{Esref Ozturk}
\IEEEauthorblockA{\textit{CENG} \\
\textit{METU}\\
Ankara, Turkey \\
esrefozturk93@gmail.com}
}

\maketitle

\begin{abstract}
This document is a model and instructions for \LaTeX.
This and the IEEEtran.cls file define the components of your paper [title, text, heads, etc.]. *CRITICAL: Do Not Use Symbols, Special Characters, Footnotes, 
or Math in Paper Title or Abstract.
\end{abstract}

\begin{IEEEkeywords}
component, formatting, style, styling, insert
\end{IEEEkeywords}

\section{Introduction}
This document is a model and instructions for \LaTeX.
Please observe the conference page limits. 

\section{Discussion}

\subsection{Effect of the Bias}

Accuracies with and without biases can be seen on the following table:

\begin{tabular}{l*{6}{c}r}
eta  & Accuracy(without bias) & Accuracy(with bias) \\
\hline
3e-4 & 0.4 & 0.48 \\
1e-3 & 0.57 & 0.58 \\
1e-1 & 0.65 & 0.79 \\
\end{tabular}

As it can be seen, bias increases the accuracy.

\subsection{Feature Scaling Technique}





\end{document}
